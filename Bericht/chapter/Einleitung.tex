\chapter{Einleitung}
\section{Einführung}

Die vorliegende Arbeit beschäftigt sich mit der Fragestellung, ob und wie ein technisch konstruiertes CAD-Modell bspw. eine Maschine, ein Auto oder wie in diesem Fall eine Windenergieanlage (WEA) in das Echtzeitsystem Unity überführt werden kann. Dazu wird untersucht welche Möglichkeiten des Exportes professionelle Konstruktionsprogramme anbieten und wie die Qualität dieser 3D-Modelle beschaffen ist. Anhand eines Beispiels werden die Zwischenschritte einer Aufbereitung beschrieben und Vergleiche mit einer Software gezogen, die diese Aufbereitung automatisch übernimmt. An diesem Beispiel soll ebenfalls das Rigging und die Animierung in Maya sowie die Erstellung eines Prototypen in Unity demonstriert werden. 

\newpage
\section{Motivation}

Diese Arbeit beschäftigt sich mit der Fragestellung, ob und wie ein technisch konstruiertes CAD-Modell bspw. eine Maschine, ein Auto oder wie in diesem Fall eine Windenergieanlage in das Echtzeitsystem Unity überführt werden kann. Dazu wird untersucht welche Möglichkeiten des Exportes professionelle Konstruktionsprogramme anbieten und wie die Qualität dieser 3D-Modelle beschaffen ist. Anhand eines Beispiels werden die Zwischenschritte einer Aufbereitung beschrieben und Vergleiche mit einer Software gezogen, die diese Aufbereitung automatisch übernimmt. An diesem Beispiel soll ebenfalls das Rigging und die Animierung in Maya sowie die Erstellung eines Prototypen in Unity demonstriert werden. 

\begin{acronym}
\acro{wea}[WEA]{Windenergieanlage}
\end{acronym}

\newpage