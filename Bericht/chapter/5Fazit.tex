\chapter{Fazit}
\label{sec:Fazit}

Abschließend lässt sich konstatieren, dass es mit einem gewissen Aufwand möglich ist ein CAD-Modell in eine VR-Anwendung zu überführen. Der größte Aufwand konzentrierte sich hauptsächlich auf die Aufbereitung des CAD-Modells. Die Arbeitszeit verteilte sich zu ca. 50\% auf die Aufbereitung, zu ca. 25\% auf die Implementierung in Unity und zu ca. 25\% auf den Projektbericht. Natürlich richtet sich die Aufbereitungszeit nach der Komplexität des Modells. Das in diesem Projekt verwendete CAD-Modell der WEA zeichnete sich durch eine eher geringe Komplexität aus. CAD-Modelle können theoretisch eine vielfach höhere Komplexität annehmen. 
Ein vorher untersuchtes Modell einer \glqq Blasformmaschine\grqq, zur Herstellung von Hohlkörpern wie Kanister o.Ä. schlug bei ähnlichem Detailgrad mit einer Polygonanzahl von ca. 5.000.000 zu Buche. Das entspricht einer zehnfach höheren Komplexität als die der WEA, die nur ca. 500.000 Polygone aufweist. Die Zeit zur Aufbereitung wäre dementsprechend höher. Daher ist  es durchaus angebracht automatisierte Verfahren zu verwenden. Dies kann z.B. über die neue Software PiXYZ geschehen, die auf Microsofts Entwicklerkonferenz \glqq Build 2018\grqq in Seattle vorgestellt wurde. PiXYZ wurde speziell für diesen Fall entwickelt.\footnote{Vgl. Dustin Burg im Unity Blog (2018): \textit{Unity at Microsoft Build: 7 takeaways from the show}.\newline
\url{https://blogs.unity3d.com/2018/05/11/unity-at-microsoft-build-7-takeaways-from-the-show/},\newline 
abgerufen am 12.09.2018.} Die Ergebnisse dieser Software können durchaus unter ähnlichen Kriterien in einem weiterführenden Forschungsprojekt untersucht werden. Das Video \glqq 1) Take CAD to mixed reality with Unity\grqq\, im Unity Blog (siehe Fußnote) macht bzgl. der Qualität des CAD-Imports  einen sehr guten ersten Eindruck. Generell wäre es wünschenswert den Arbeitsschritt der Retopologisierung zu automatisieren, da es sich um eine reine Fleißarbeit handelt, die wenig Kreativität erfordert. Zudem sind spezielle Kenntnisse in der Polygonmodellierung in Maya oder einem äquivalenten 3D-Programm erforderlich. Die freigewordene  Arbeitszeit kann in anspruchsvollere und kreativere Tätigkeiten, wie der Entwicklung neuer Features in Unity investiert werden.
Ferner können weitere Anwendungsfälle entwickelt und implementiert werden. In Verbindung mit weiteren Visualisierungsmethoden können alle möglichen Fakten vermittelt werden. Folgende Umsetzungen wären denkbar:
\begin{itemize}
\item \textbf{Erzeugte Leistung:} Im Betriebszustand kann anhand einer Leistungsanzeige, einer Glühlampe o.Ä. die erzeugte Energie visualisiert werden. Im Wartungszustand würde z.B. keine Energie erzeugt werden, was diese Anzeige verdeutlichen könnte.
 
\item \textbf{Induktion nach dem Generatorprinzip:} Wie aus mechanischer Energie elektrische wird kann im Generator der WEA durch Darstellung von rotierenden Spulwicklungen in einem Magnetfeld sowie durch entsprechende sinusförmige Verläufe, ähnlich einem Oszillograph visualisiert werden.
 
\item \textbf{Manipulation der Windrichtung:} Die Beeinflussung der Windrichtung und die korrekte Reaktion der Windrichtungsnachführung auf die Änderung kann visuell am animierten Modell dargestellt werden.

\item \textbf{Darstellung von Strömungen:} Anhand von Partikelemittern kann das Strömungsverhalten der Rotorgeometrie bzw. Rotorblattgeometrie simuliert und evaluiert werden. Dies setzt natürlich die Implementierung eines physikalischen Modells aus dem Bereich der Strömungslehre voraus. 
 
\end{itemize} 

Dies sind nur vier mögliche Anwendungsfälle. Natürlich muss immer geprüft werden, was die Anforderungen an das System sind. So kann man sich die Frage stellen, ob der Aufwand der Implementierung eines physikalischen Modells aus der Strömungslehre tatsächlich Sinn macht. Schließlich sollte das Strömungsverhalten einer WEA oder bspw. eines Flugzeuges schon digital während der  Konstruktionsphase oder anhand eines Kunstoff- oder Holzmodells in einem Windkanal getestet werden. 
